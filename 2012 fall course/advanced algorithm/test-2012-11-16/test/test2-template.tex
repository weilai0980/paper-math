\documentclass[a4paper,11pt]{book}
%
% HOW TO USE THIS TEMPLATE:
%
% IN THE TWO INDICATED PLACES BELOW,
%   ENTER YOUR NAME
%
% FOR EACH PROBLEM IN THE SET, WRITE YOUR SOLUTION IN A FILE
%   NAMED "Problem1.tex", "Problem2.tex", etc.
%   THESE SOLUTIONS FILES HAVE NO PREAMBLE AND
%   NO \begin{document} \end{document}
%   (THEY JUST GET INLINED WHEN RUNNING LaTeX)
%
% PLACE THESE SOLUTIONS FILES AND THE TEMPLATE IN THE SAME DIRECTORY
%
% RUN LaTeX
%
% IF YOU NEED TO ADD PACKAGES (E.G., FOR HANDLING FIGURES, OR
%   FOR RUNNING PDFLaTeX), PUT IN A NEW \usepackage LINE AS
%   SHOWN BELOW

%
%%%% REPLACE Robert Endre Tarjan WITH YOUR COMPLETE NAME
%%%% NOTE: IF YOUR LAST NAME HAS ACCENTS, SOME LaTeX COMMANDS ARE:
%%%%       for diaeresis/umlaut ", as in German, use \"a, \"u, \"o, etc.
%%%%       for grave accent `, use \`e, \`a, \`u
%%%%       for acute accent ', use \'a, \'e, \'u
%%%%       for tilde ~, use \tilde{a}, \tilde{o}
%%%%       for caret ^, use \hat{e}, \hat{o}
%%%%	   if you need an accent on an i, say an umlaut, use \"\i{}
%%%%		the \i{} is an i without its dot; e.g., na\"\i{}ve
%%%%	   these all work well with lower-case letters,
%%%%	        but only so-so with upper-case letters
%%%%
  \def\myname{Robert Endre Tarjan}

%%%% IF ADDITIONAL PACKAGES ARE NEEDED, UNCOMMENT LINE BELOW
%%%% AND ENTER THE PACKAGE NAMES
%  \usepackage{additional_package_1,additional_package_2}



% DO NOT MODIFY THIS NEXT PART
%%%%%
  \usepackage{times,amsmath,pslatex,graphicx}
  \newcounter{problem}
  \setcounter{problem}{0}
  \makeatletter
  \def\ps@headings{%
    \let\@mkboth\markboth
    \def\@evenfoot{\hfil\large\sf Test 2, Problem \theproblem\hfil}
    \def\@oddfoot{\@evenfoot}
    \def\@evenhead{\large\sc\myname\hfil\it\today\hfil\sf P.~\thepage}
    \def\@oddhead{\@evenhead}}
  \makeatother
  \pagestyle{headings}
  \advance\textwidth by16mm
  \advance\oddsidemargin by-8mm
  \advance\textheight by10mm
%%%%%

\begin{document}
\begin{center}
  \LARGE\sf
  \begin{tabular}{||c|c|c|c||}
    \hline\hline
    Prob.~1 & Prob.~2 & Prob.~3 & Prob.~4 \\
    \hline
    &&&\\
    \hline\hline
  \end{tabular}
\end{center}
\bigskip\bigskip

%%%%%%%%%%%%%%%%%%%%%%%%%%%%%%%%%%%%%%%%%%%%%%%%%%%%%%%%%%%%
\addtocounter{problem}{1}
\hbox to\textwidth{\huge\sf
  Problem \theproblem .\hfil \framebox{\hbox to2cm{\hfil /25}}}

\bigskip\noindent
Recall that the subgraph $G'$ of $G$ induced by a subset $V'$ of $V$ is
the graph $G'=(V',E')$, where $E'$ is the subset of $E$ containing exactly
those edges with both endpoints in $V'$.
You are given an undirected graph $G=(V, E)$ (as an array of adjacency lists)
and a positive integer $K$
and asked to find a maximum subset of vertices such that the subgraph
of $G$ induced by these vertices has degree at least $K$ (i.e., every vertex
in the induced subgraph has degree of at least $K$).

Consider the following greedy algorithm:
\begin{quote}
  \sf
  repeatedly remove the remaining vertex of smallest (updated) degree,
  until all remaining vertices have degree at least $K$.
\end{quote}
We use a priority queue to retrieve the remaining vertex of smallest degree;
since removing a vertex decreases the degree of its neighbors, we use a
Fibonacci heap to take advantage of its efficient \texttt{DecreaseKey}
operation.  Thus our algorithm begins by building a Fibonacci heap containing
all vertices, using the degree of each vertex as the priority key.
Our main loop then simply runs a \texttt{DeleteMin} operation on the heap.
If the heap is empty, the algorithm stops and returns the empty set;
otherwise, it tests the key (degree) of the returned vertex $v$ against $K$.
If the degree is at least $K$, the algorithm stops and returns the collection
of vertices still in the heap.  Otherwise, the algorithm updates the degrees
of the neighbors of $v$ (listed in the adjacency list of $v$) using
\texttt{DecreaseKey} and begins the next iteration.

Prove the correctness of this algorithm and that it runs in
$O(|E|+|V|\cdot\log(|V|))$ worst-case time.


\medskip
\noindent
\textbf{Solution}

\vspace{5mm}
\noindent
\textbf{Proof of correctness}

The proposed greedy method transforms the original objective function that maximizes the subset of vertices with at least $K$ degree to one that minimize the subset of vertices such that the sub-graph induced by these vertices has degree smaller than $K$. Therefore, the left vertices construct the maximal sub-graph of G that includes the vertices with at least $K$ degree. 

I will use the ``cut and paste'' method to prove that the removed vertices by the greedy method constitute the minimal subset so that the induced sub-graph the vertices of which all have degree smaller than $K$. 

Suppose there is an optimal solution, a vertex sub-set $V^*$ to be removed. $ V^*= {v_1^*,v_2^*, \cdots, v_n^* }$ are the sequential deleted vertices with degree smaller than $K$ and the size of this subset is $N$.

The function $d(v)$ represents the degree of vertex $v$. According to the proposed greedy algorithm, initially the vertex with smallest degree is denoted as $v_0$ and its degree is smaller than $K$(Otherwise, the greedy algorithm won't run), hence $d(v_1^*) \leq d(v_0) $.

Case 1. If $v_0$ is the $v_i^*$ in $V^*$, namely, $V^*=\{ v_1^*,v_2^*, \cdots, v_{i-1}^* ,v_i^*, \cdots, v_n^* \}$. 

During the sequential deletion of $v_1^*,v_2^*, \cdots, v_{i-1}^*$, the degree of $v_i^*$ may be updated because some of vertices in $\{ v_1^*,v_2^*, \cdots, v_{i-1}^* \}$ have edges linking $v_i^*$. Also, the degree of $\{ v_{i+1}^*, \cdots, v_n^* \}$ will be updated because of the deletion of $v_i^*$. 

If I exchange the position of $v_1^*$ and $v_i^*$, the degree of all vertices in $V^*/v_i^*$ connecting $v_i^*$ would be updated after the $v_i^*$ is removed first. Moreover, these vertices' degree would be the same as the those in vertex removing process of original $V^*$ where $v_i^*$ is the $i$-th deleted. Therefore, the optimal solution $V^*$ can be transformed to anther optimal solution with the first greedy choice.

Case 2. When $v_0$ is not in $V^*$, the $v_1^*$ is replaced by $v_0$ and $V_\alpha= V^*- v_1^* \bigcap v_0 $, we will see if $V_\alpha$ still hold the optimality, namely, $|V_\alpha| = V^*$.

Here, I define another function, $R(v)$ stands for the a set of vertices that have edges with $v$ as one of end-points.

$r(v)$ is a sub-set of $R(v)$ including the vertices both in $R(v)$ and $V^*$, namely, $r(v)= R(v) \cap V^*$.

If $R(v_1^*)) \cap  R(v_0) = \phi $,  since the first deleted vertex in $V_\alpha$ is $v_0$, the vertices in $r(v_1^*)$'s degree won't updated. Therefore, the degree of some vertices in $r(v_1^*)$ may be larger than $K$. In $V^*$, we denote the set of  vertices whose deletions result from removing $v_1^*$ as set $\xi$. 

Some vertices' degree may be decreased by 1 as the deletion of $v_0$ and smaller than $K$. These vertices should be added to $V_\alpha$. 
 
If $R(v_1^*)) \cap  R(v_0) \neq \phi $
 
 

\vspace{5mm}
\noindent
\textbf{Time complexity}

%%%%%%%%%%%%%%%%%%%%%%%%%%%%%%%%%%%%%%%%%%%%%%%%%%%%%%%%%%%%
\chapter*{}
\addtocounter{problem}{1}
\hbox to\textwidth{\huge\sf
  Problem \theproblem .\hfil \framebox{\hbox to2cm{\hfil /25}}}

\bigskip\noindent
Give a polynomial-time algorithm for each of the following problems---sketch
a proof of their correctness and analyze their running time.

You are given an undirected graph $G=(V,E)$ with (not necessarily distinct)
positive integral weights for the edges, and an edge $e_0\in E$.
\begin{enumerate}
  \itemsep 1pt
  \item
    Decide whether \emph{every} minimum spanning tree of $G$ contains $e_0$.
  \item
    Decide whether \emph{some} minimum spanning tree of $G$ contains $e_0$.
\end{enumerate}

\medskip
\noindent
\textbf{Solution}

\noindent
\textbf{(1)}

Case 1. The weights of all edges are distinct.

Under this situation, $G$ has unique MST. Therefore, we can just use  

Case 2. The weights of all edges are not distinct.




\noindent
\textbf{(2)}

%%%%%%%%%%%%%%%%%%%%%%%%%%%%%%%%%%%%%%%%%%%%%%%%%%%%%%%%%%%%
\chapter*{}
\addtocounter{problem}{1}
\hbox to\textwidth{\huge\sf
  Problem \theproblem .\hfil \framebox{\hbox to2cm{\hfil /25}}}

\bigskip\noindent
Give a polynomial-time algorithm for each of the following problems---sketch
a proof of their correctness and analyze their running time.

You are given a bipartite graph $G=(V_1, V_2, E)$, $|V_1|=|V_2|$, and an
edge $e_0\in E$.
\begin{enumerate}
  \itemsep 1pt
  \item
    Decide whether \emph{every} perfect matching of $G$ contains $e_0$.
  \item
    Decide whether \emph{some} perfect matching of $G$ contains $e_0$.
\end{enumerate}

\medskip
\input{problem3.tex}

%%%%%%%%%%%%%%%%%%%%%%%%%%%%%%%%%%%%%%%%%%%%%%%%%%%%%%%%%%%%
\chapter*{}
\addtocounter{problem}{1}
\hbox to\textwidth{\huge\sf
  Problem \theproblem .\hfil \framebox{\hbox to2cm{\hfil /25}}}

\bigskip\noindent
You are given an undirected network $N=(V, E)$ with integral edge capacities,
a source, $s$, and a sink, $t$.
Define $N$ to be \emph{$k$-stable} if and only if the value of the maximum
$s$-$t$ flow does not increase under any $k$-edge alteration.  A $k$-edge
alteration consists of picking $k$ edges of the network and assigning them
arbitrary new (positive integral) capacities.
\begin{enumerate}
  \item
    Design an algorithm to test whether $N$ is $1$-stable; your
    algorithm should run in $O(|V|^2 \cdot |E|)$ time.
  \item
    Design an algorithm to test whether $N$ is $2$-stable; your
    algorithm should also run in $O(|V|^2 \cdot |E|)$ time,
    although it is likely to involve significantly more work.
\end{enumerate}

\medskip
\input{problem4.tex}
\end{document}
